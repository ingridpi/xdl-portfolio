\chapter{Code} \label{app:code}

The following section illustrates the code implementation of the project. It includes the main components of the codebase such as data download and pre-processing, model training and evaluation, benchmarking and explainability techniques. 

The code is structured in a modular way, where each module corresponds to a specific functionality. The main components are:
\begin{itemize}
    \item \textbf{Configuration}: This module manages the configuration settings that define the list of assets, the training-test periods, and other hyper-parameters.
    \item \textbf{Data Download and Pre-processing:} This module handles the downloading of datasets, cleaning, and pre-processing steps necessary for training the models.
    \item \textbf{Agents}: This module contains the implementation of the \acrshort{drl} agents capable of interacting with the environment, learning from it, and making decisions based on the observations received.
    \item \textbf{Environments}: This module defines the various environments in which the agents operate, including the state and action spaces, as well as the reward functions.
    \item \textbf{Benchmarking}: This module contains the implementation of the benchmarking strategies as well as the evaluation metrics used to assess the performance of the agents.
    \item \textbf{Explainability}: This module implements the explainability techniques used to interpret the agents' decisions and understand their behaviour.
    \item \textbf{Visualiser}: This module provides visualisations tools for the data, agents' performance, benchmarking comparisons and explainability results.
    \item \textbf{Optimisation}: This module contains the integration with hyper-parameter tuning libraries to find optimal configurations and track the experiments.
\end{itemize}

The modules are implemented as Python packages, each containing the necessary code and resources for its specific functionality. Additionally, the project contains a folder titled \texttt{examples}, where the main Jupyter notebooks to perform each of the separate tasks of this project are implemented. The directory structure is organised as follows:

% Add caption and smaller text size and less spacing between lines
\begin{verbatim}
xdl-portfolio/
|
+-- agents/
|   +-- __init__.py
|   +-- drl_agent.py
|   +-- tb_callback.py
|
+-- config/
|   +-- __init__.py
|   +-- config.py
|   +-- config_indicators.py
|   +-- config_tickers.py
|   +-- config_models.py
|
+-- environments/
|   +-- __init__.py
|   +-- env_portfolio_optimisation.py
|
+-- examples/
|   +-- findownloader.ipynb
|   +-- finpreprocessor.ipynb
|   +-- portfolio_optimisation.ipynb
|   +-- backtesting.ipynb
|   +-- hyperparameter_tuning.ipynb
|   +-- explainability.ipynb
|
+-- explainability/
|   +-- __init__.py
|   +-- explainability.py
|   +-- feature_importance.py
|   +-- lime_explainability.py
|   +-- shap_explainability.py
|
+-- optimisation/
|   +-- __init__.py
|   +-- wandb_opt.py
|
+-- pbenchmark/
|   +-- __init__.py
|   +-- portfolio_benchmark.py
|
+-- preprocessor/
|   +-- __init__.py
|   +-- findata_downloader.py
|   +-- findata_preprocessor.py
|
+-- visualisation/
|   +-- __init__.py
|   +-- benchmark_visualiser.py
|   +-- findata_visualiser.py
|   +-- model_visualiser.py
|   +-- shap_visualiser.py
|   +-- style.mplstyle
+-- README.md
+-- requirements.txt
\end{verbatim}

The code is available as a GitHub repository at \href{https://github.com/ingridpi/xdl-portfolio.git}{XDL-Portfolio}.

\section{Configuration Module}

\captionof{listing}{\texttt{config.py}}
\inputminted[fontsize=\footnotesize,linenos,breaklines=true,breakanywhere=true,frame=lines,baselinestretch=1]{python}{../config/config.py}

\captionof{listing}{\texttt{config\_tickers.py}}
\inputminted[fontsize=\footnotesize,linenos,breaklines=true,breakanywhere=true,frame=lines,baselinestretch=1]{python}{../config/config_tickers.py}

\captionof{listing}{\texttt{config\_indicators.py}}
\inputminted[fontsize=\footnotesize,linenos,breaklines=true,breakanywhere=true,frame=lines,baselinestretch=1]{python}{../config/config_indicators.py}

\captionof{listing}{\texttt{config\_models.py}}
\inputminted[fontsize=\footnotesize,linenos,breaklines=true,breakanywhere=true,frame=lines,baselinestretch=1]{python}{../config/config_models.py}

\section{Data Download and Pre-processing Module}

\captionof{listing}{\texttt{findata\_downloader.py}}
\inputminted[fontsize=\footnotesize,linenos,breaklines=true,breakanywhere=true,frame=lines,baselinestretch=1]{python}{../preprocessor/findata_downloader.py}

\captionof{listing}{\texttt{findata\_preprocessor.py}}
\inputminted[fontsize=\footnotesize,linenos,breaklines=true,breakanywhere=true,frame=lines,baselinestretch=1]{python}{../preprocessor/findata_preprocessor.py}

\section{Agents Module}

\captionof{listing}{\texttt{drl\_agent.py}}
\inputminted[fontsize=\footnotesize,linenos,breaklines=true,breakanywhere=true,frame=lines,baselinestretch=1]{python}{../agents/drl_agent.py}

\captionof{listing}{\texttt{tb\_callback.py}}
\inputminted[fontsize=\footnotesize,linenos,breaklines=true,breakanywhere=true,frame=lines,baselinestretch=1]{python}{../agents/tb_callback.py}

\section{Environments Module}

\captionof{listing}{\texttt{env\_portfolio\_optimisation.py}}
\inputminted[fontsize=\footnotesize,linenos,breaklines=true,breakanywhere=true,frame=lines,baselinestretch=1]{python}{../environments/env_portfolio_optimisation.py}

\section{Benchmarking Module}

\captionof{listing}{\texttt{portfolio\_benchmark.py}}
\inputminted[fontsize=\footnotesize,linenos,breaklines=true,breakanywhere=true,frame=lines,baselinestretch=1]{python}{../pbenchmark/portfolio_benchmark.py}

\section{Explainability Module}

\captionof{listing}{\texttt{explainability.py}}
\inputminted[fontsize=\footnotesize,linenos,breaklines=true,breakanywhere=true,frame=lines,baselinestretch=1]{python}{../explainability/explainability.py}

\captionof{listing}{\texttt{feature\_importance.py}}
\inputminted[fontsize=\footnotesize,linenos,breaklines=true,breakanywhere=true,frame=lines,baselinestretch=1]{python}{../explainability/feature_importance.py}

\captionof{listing}{\texttt{lime\_explainability.py}}
\inputminted[fontsize=\footnotesize,linenos,breaklines=true,breakanywhere=true,frame=lines,baselinestretch=1]{python}{../explainability/lime_explainability.py}

\captionof{listing}{\texttt{shap\_explainability.py}}
\inputminted[fontsize=\footnotesize,linenos,breaklines=true,breakanywhere=true,frame=lines,baselinestretch=1]{python}{../explainability/shap_explainability.py}

\section{Visualisation Module}

\captionof{listing}{\texttt{\_\_init\_\_.py}}
\inputminted[fontsize=\footnotesize,linenos,breaklines=true,breakanywhere=true,frame=lines,baselinestretch=1]{python}{../visualiser/__init__.py}

\captionof{listing}{\texttt{benchmark\_visualiser.py}}
\inputminted[fontsize=\footnotesize,linenos,breaklines=true,breakanywhere=true,frame=lines,baselinestretch=1]{python}{../visualiser/benchmark_visualiser.py}

\captionof{listing}{\texttt{findata\_visualiser.py}}
\inputminted[fontsize=\footnotesize,linenos,breaklines=true,breakanywhere=true,frame=lines,baselinestretch=1]{python}{../visualiser/findata_visualiser.py}

\captionof{listing}{\texttt{model\_visualiser.py}}
\inputminted[fontsize=\footnotesize,linenos,breaklines=true,breakanywhere=true,frame=lines,baselinestretch=1]{python}{../visualiser/model_visualiser.py}

\captionof{listing}{\texttt{shap\_visualiser.py}}
\inputminted[fontsize=\footnotesize,linenos,breaklines=true,breakanywhere=true,frame=lines,baselinestretch=1]{python}{../visualiser/shap_visualiser.py}

\captionof{listing}{\texttt{style.mplstyle}}
\inputminted[fontsize=\footnotesize,linenos,breaklines=true,breakanywhere=true,frame=lines,baselinestretch=1]{python}{../visualiser/style.mplstyle}

\section{Optimisation Module}

\captionof{listing}{\texttt{wandb\_opt.py}}
\inputminted[fontsize=\footnotesize,linenos,breaklines=true,breakanywhere=true,frame=lines,baselinestretch=1]{python}{../optimisation/wandb_opt.py}

\section{Examples}

The examples are originally a set of Jupyter Notebooks. However, for visualisation purposes in this report, they have been converted to Python scripts. 

\texttt{findownloader.ipynb} demonstrates how to download the data for the tickers specified in the \texttt{config/config.py} file.
\captionof{listing}{\texttt{findownloader.py}}
\inputminted[fontsize=\footnotesize,linenos,breaklines=true,breakanywhere=true,frame=lines,baselinestretch=1]{python}{../examples/findownloader.py}

\texttt{finpreprocessor.ipynb} demonstrates how to preprocess the data downloaded in the previous step. This includes calculating technical indicators, macroeconomic indicators, and other features.
\captionof{listing}{\texttt{finpreprocessor.py}}
\inputminted[fontsize=\footnotesize,linenos,breaklines=true,breakanywhere=true,frame=lines,baselinestretch=1]{python}{../examples/finpreprocessor.py}

\texttt{portfolio\_optimisation.ipynb} demonstrates how to use the portfolio optimisation environment and train agents to optimise a portfolio of stocks using reinforcement learning.
\captionof{listing}{\texttt{portfolio\_optimisation.py}}
\inputminted[fontsize=\footnotesize,linenos,breaklines=true,breakanywhere=true,frame=lines,baselinestretch=1]{python}{../examples/portfolio_optimisation.py}

\texttt{backtesting.ipynb} demonstrates how to run the benchmark strategies and compare the performance of the trained agents against these benchmarks according to the evaluation metrics.
\captionof{listing}{\texttt{backtesting.py}}
\inputminted[fontsize=\footnotesize,linenos,breaklines=true,breakanywhere=true,frame=lines,baselinestretch=1]{python}{../examples/backtesting.py}

\texttt{hyperparameter\_tuning.ipynb} demonstrates how to use \href{https://wandb.ai/site/}{Weights and Biases} to perform hyper-parameter tuning by running a sweep, whose configuration is in \texttt{config.py} and \texttt{config\_models.py}.
\captionof{listing}{\texttt{hyperparameter\_tuning.py}}
\inputminted[fontsize=\footnotesize,linenos,breaklines=true,breakanywhere=true,frame=lines,baselinestretch=1]{python}{../examples/hyperparameter_tuning.py}

\texttt{explainability.ipynb} demonstrates how to use the explainability tools (feature importance, LIME and SHAP) to understand the decision-making process of a single model.
\captionof{listing}{\texttt{explainability.py}}
\inputminted[fontsize=\footnotesize,linenos,breaklines=true,breakanywhere=true,frame=lines,baselinestretch=1]{python}{../examples/explainability.py}
