\chapter{Conclusion} \label{ch:conclusion}

This thesis has explored the application of \acrfull{drl} algorithms for optimal portfolio allocation in dynamic financial markets. The primary objective was to develop an explainable model-agnostic framework capable of enhancing the understanding of any \acrshort{drl} algorithm predictions, with the goal of providing insights into the decision-making process of these complex models as well as being a tool for auditability purposes. 

Five state-of-the-art \acrshort{drl} algorithms were implemented and evaluated on a portfolio management task. To assess the behaviour of these algorithms in different market conditions, a comprehensive experimental setup was designed, involving various datasets and environment representations. The datasets ranged from equities to commodities and currencies, each presenting unique challenges and opportunities for the \acrshort{drl} algorithms. The results demonstrated that \acrshort{drl} algorithms can effectively learn and adapt to dynamic market conditions, achieving competitive performance compared to traditional portfolio management strategies. However, the challenge remains in finding the optimal hyper-parameters, uniquely suited to each algorithm and dataset combination, in order to be able to fully exploit their potential.

Regarding the explainability aspect, the framework developed in this thesis successfully enhances the interpretability, transparency and auditability of these \gls{blackbox} models. The incorporation of feature importance through a surrogate model, \acrfull{lime} analysis and \acrfull{shap} values provides an exhaustive methodology for understanding both individual predictions and the overall decision-making process of the models over a test set. However, the results highlight the superiority of the \acrshort{shap} technique, which is capable of providing global explanations and feature importance without the need for an additional layer, in conjunction with local explanations. 
