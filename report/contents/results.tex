\chapter{Results} \label{ch:results}

This chapter presents the results of conducting experiments under the methodology proposed in Chapter \ref{ch:methodology}. The experiments were designed to evaluate the performance of the implemented \acrshort{drl} models for portfolio optimisation in changing environment representations and market conditions. Moreover, to analyse the interpretability of the model's decisions, a framework using post-hoc explainability techniques is explored.

\section{Dataset} \label{sec:dataset}

Given the general difficulty in finding the appropriate \acrshort{drl} algorithm with a suitable \gls{rewardfunction} for portfolio optimisation, the five implemented algorithms were tested on five different datasets. Each dataset consists of a different set of financial assets, ranging from three different asset classes. First, three datasets were constructed using the stock constituents of three renowned indexes:
\begin{itemize}
    \item \acrfull{djia} with 30 stocks,
    \item \acrfull{eurostoxx50} with 50 stocks,
    \item \acrfull{ftse100} with 100 stocks.
\end{itemize}

The constituents of each of the indexes were retrieved in April 2025 and can be found in Appendix \ref{sec:datasets-equities}. It is important to note that the datasets were chosen to illustrate different currencies, as this introduces another factor of changing market conditions. 

Additionally, two datasets were constructed using commodities and currencies, respectively. The commodities dataset includes six different commodities, which are listed in Appendix \ref{sec:datasets-commodities}. These are a sample of the most traded commodities in the market and were chosen by their availability in the \texttt{Yahoo! Finance API} \footnote{https://uk.finance.yahoo.com}. With regard to the currencies dataset, it includes ten different currency pairs, listed in Appendix \ref{sec:datasets-currencies}. These were selected based on their trading volume and liquidity, with all pairs quoted in \acrfull{usd}.

The datasets are constructed using daily data from January 2016 to July 2025 downloaded using the Python \texttt{yfinance} library \cite{yfinance}. The dataset is partitioned into two disjoint sets: training and testing, with the training set containing data from January 2016 to December 2023, and the testing set starting on January 2024 until July 2025. The training set is used to train the \acrshort{drl} models, while the testing set is used to evaluate their performance. For hyper-parameter tuning, the training set is further split into a training and validation set, with the validation set corresponding to the period between January 2023 and December 2023. The validation set is used to evaluate the performance of the models for each hyper-parameter combination. The train-validation-test split is summarised in Table \ref{tab:dataset-split}.

\begin{longtable}{|l|l|l|l|}
    \hline
    \textbf{Dataset} & \textbf{Training Period} & \textbf{Validation Period} & \textbf{Testing Period} \\ \midrule
    \endfirsthead

    \hline
    \textbf{Dataset} & \textbf{Training Period} & \textbf{Validation Period} & \textbf{Testing Period} \\ \midrule
    \endhead

    \endfoot
    \hline
    DJIA & Jan 2016 - Dec 2022 & Jan 2023 - Dec 2023 & Jan 2024 - Jul 2025 \\
    \hline
    Euro Stoxx 50 & Jan 2016 - Dec 2023 & Not applicable & Jan 2024 - Jul 2025 \\
    \hline
    FTSE 100 & Jan 2016 - Dec 2023 & Not applicable & Jan 2024 - Jul 2025 \\
    \hline
    Commodities & Jan 2016 - Dec 2022 & Jan 2023 - Dec 2023 & Jan 2024 - Jul 2025 \\
    \hline
    Currencies & Jan 2016 - Dec 2022 & Jan 2023 - Dec 2023 & Jan 2024 - Jul 2025 \\
    \hline
    \caption{Train-Validation-Test Split for each dataset}
    \label{tab:dataset-split}
\end{longtable}

\section{Experiment Design} \label{sec:experiment-design}

To address the challenge of finding a suitable algorithm for portfolio optimisation, the five implemented \acrshort{drl} algorithms were tested on the five datasets described in Section \ref{sec:dataset}, with the goal of evaluating the performance of each algorithm in different scenarios and market conditions. Moreover, the environment representation will also be varied to assess the impact of more information on the model's performance. Four environment representations were considered, each with a different number of features:
\begin{itemize}
    \item Simple dataset: \acrfull{ohlcv} prices of the assets.
    \item Covariance dataset: To the simple dataset, the covariance matrix of the assets is added to explicitly model the relationships between the assets.
    \item Indicators dataset: Technical and macroeconomic indicators are added to the simple dataset.
    \item Complete dataset: The complete dataset includes the simple dataset, the covariance matrix and the technical and macroeconomic indicators.
\end{itemize}

The strength of \acrshort{drl} algorithms lies in their ability to learn from high-dimensional data, which is why the goal is to evaluate whether a more exhaustive environment representation leads to better performance. However, with higher dimensionality comes a higher computational cost.

Another particular challenge is the choice of suitable reward function, which is crucial for the success of the algorithms in any \acrshort{rl} setting. The reward function should ideally be designed to encourage the model to learn an investment strategy that maximises returns while minimising risk. As a result, two choices of reward function were considered:
\begin{itemize}
    \item Change in portfolio value: The reward is the change in portfolio value at each time step, which encourages the model to maximise returns.
    \item Sharpe ratio: The reward is the Sharpe ratio of the portfolio, which measures the risk-adjusted return.
\end{itemize}

Finally, the performance of the algorithms is closely related to the choice of hyper-parameters. As such, the hyper-parameters were tuned using Bayesian optimisation, to find the optimal configuration for each algorithm and dataset combination.

Overall, the experiments were designed to evaluate the performance of the implemented \acrshort{drl} algorithms in different scenarios, with the goal of finding the most suitable algorithm for portfolio optimisation. However, testing 5 algorithms on 5 distinct datasets with 4 possible environment representations and 2 potential reward functions would result in a total of 40 different experiments. Additionally, optimising the parameters for each experiment further expands the experimental space and significantly increases the computational time required. Due to limited computational resources\footnote{The university did not provide access to a computing cluster; therefore, all experiments were conducted on a personal computer.}, the scope of experiments was adjusted as follows:
\begin{itemize}
    \item Hyper-parameter tuning was performed only for the Dow Jones 30 dataset with simple and indicators environment representation, as it is the smallest equities dataset and requires less computational time.
    \item Since the covariance matrix, significantly increases the dimensionality of the environment representation, it was only included in the experiments with the Dow Jones 30, the currencies and the commodities datasets.
    \item The reward function was set to the change in portfolio value for all experiments, as it is the most straightforward and intuitive choice for portfolio optimisation. However, the Sharpe ratio performance was evaluated in the Dow Jones 30 dataset, with environment representations that exclude the covariance.
\end{itemize}

The final experimental design consists of 18 experiments, which are summarised in Table \ref{tab:experiments-summary}, where each row represents a unique combination of dataset, environment representation, reward function and whether hyper-parameter tuning is performed for this combination.

\begin{longtable}{|p{3cm}|p{3cm}|p{3.5cm}|p{3.5cm}|}
    \hline
    \textbf{Dataset} & \textbf{Environment Representation} & \textbf{Hyper-parameter Tuning} \\ \midrule
    \endfirsthead

    \hline
    \textbf{Dataset} & \textbf{Environment Representation} & \textbf{Hyper-parameter Tuning} \\ \midrule
    \endhead

    \endfoot
    \hline
    Dow Jones 30    & Simple         & Yes \\ \hline
    Dow Jones 30    & Covariance     & No \\ \hline
    Dow Jones 30    & Indicators     & Yes \\ \hline
    Dow Jones 30    & Complete       & No \\ \hline
    Euro Stoxx 50   & Simple         & No  \\ \hline
    Euro Stoxx 50   & Indicators     & No  \\ \hline
    FTSE 100        & Simple         & No  \\ \hline
    FTSE 100        & Indicators     & No  \\ \hline
    Commodities     & Simple         & No  \\ \hline
    Commodities     & Covariance     & No  \\ \hline
    Commodities     & Indicators     & No  \\ \hline
    Commodities     & Complete       & No  \\ \hline
    Currencies      & Simple         & No  \\ \hline
    Currencies      & Covariance     & No  \\ \hline
    Currencies      & Indicators     & No  \\ \hline
    Currencies      & Complete       & No  \\ \hline
\caption{Summary of experiments conducted.}
\label{tab:experiments-summary}
\end{longtable}


\section{Evaluation} \label{sec:evaluation}

As outlined in the previous section, the experiments are designed to evaluate the performance of the implemented \acrshort{drl} algorithms in different scenarios and market conditions. The evaluation will focus on key performance metrics, as well as benchmarking against traditional portfolio optimisation techniques.

\subsection{Performance Metrics} \label{sec:performance-metrics}

The performance metrics are provided through the \texttt{pyfolio} library \cite{pyfolio}, which includes a \texttt{perf\_stats} method to calculate various performance metrics of a strategy. 

The main metrics for comparison are:
\begin{itemize}
    \item The cumulative return is the total change in investment price over a period of time, representing the overall percentage gain or loss from the initial investment value. The formula is given by:
    \begin{equation}
        \text{Cumulative return} = \frac{\text{Final portfolio value} - \text{Initial portfolio value}}{\text{Initial portfolio value}}.
    \end{equation}
    \item The annualised return is the geometric average of the amount of money earned by an investment each year over a given period of time, providing a standardised measure of annual performance. It is calculated as follows:
    \begin{equation}
        \text{Annualised Return} = \left(\frac{\text{Final portfolio value}}{\text{Initial portfolio value}}\right)^{\frac{1}{\text{Number of years}}} - 1.
    \end{equation}
    \item The annualised volatility is the standard deviation of returns annualised to provide a measure of investment risk on a yearly basis and can be computed with the following formula:
    \begin{equation}
        \text{Annualised Volatility} = \text{Standard Deviation of Returns} \times \sqrt{\text{Yearly trading days}},
    \end{equation}
    where the number of trading days per year is typically assumed to be 252.
    \item The Sharpe ratio is a measure of risk-adjusted performance that compares the excess return of an investment to a risk-free asset against its volatility. The ratio is given by:
    \begin{equation}
        \text{Sharpe Ratio} = \frac{R_p - R_f}{\sigma_p},
    \end{equation}
    where $R_p$ is the annualised return of the portfolio, $R_f$ is the annualised risk-free rate, and $\sigma_p$ is the annualised volatility of the portfolio.
    \item The max drawdown is the maximum percentage loss from a peak to a trough during a specified period, indicating the worst-case scenario for portfolio decline. Its formula is:
    \begin{equation}
        \text{Max Drawdown} = \frac{\text{Peak Value} - \text{Trough Value}}{\text{Peak Value}}.
    \end{equation}
\end{itemize}

Other metrics available through the \texttt{pyfolio} library are outlined in Appendix \ref{app:evaluation_metrics}.

\subsection{Benchmark Strategies} \label{sec:benchmark-strategies}

Aside from computing relevant performance metrics, the algorithms will be benchmarked against traditional portfolio optimisation methods. The benchmarks are designed to provide a baseline for comparison and to evaluate the performance of the \acrshort{drl} algorithms in relation to established methods. The following benchmark strategies were considered.
\begin{itemize}
    \item Equal-weighted portfolio: A simple strategy that allocates an equal weight to each asset in the portfolio.
    \item Mean-variance optimisation: A classic portfolio optimisation method that aims to maximise Sharpe ratio.
    \item Min-variance portfolio: Another classic portfolio optimisation method that seeks to minimise the portfolio's volatility.
    \item Momentum portfolio: A strategy that invests in assets with positive momentum, i.e. those that have performed well in the previous time step, and avoids those with negative momentum.
\end{itemize}

The implementation of the mean-variance and the min-variance portfolio allocation strategies has been done using the \texttt{PyPortfolioOpt} Python library \cite{Martin2021}, whereas the equal-weighted and momentum strategies have been implemented using custom code.

Finally, if the portfolio is made up of equities of a relevant index, the benchmark will also include the index itself, which serves as a reference point for the performance of the portfolio. 

\section{Experiment: Algorithm Comparison} \label{sec:exp-algorithm-comparison}

In this section, the results of the experiment to identify the suitability of the implemented \acrshort{drl} algorithms for portfolio optimisation under different market conditions are presented. The algorithms are trained on data with a simple environment representation, which includes the \acrshort{ohlcv} prices of the assets, and evaluated on the 5 datasets. The table \ref{tab:experiment_algorithms_a2c} summarises the results of the experiment for the \acrshort{a2c} algorithm, where each row corresponds to a different dataset and each column to a different performance metric. The results for the other algorithms are presented in Appendix \ref{app:experiment_algorithms_comparison}.

\begin{longtable}{|l|p{2.1cm}|p{2.1cm}|p{2.1cm}|p{1.5cm}|p{2cm}|}
    \caption{Algorithm comparison results for the A2C implementation across the different datasets under the indicators feature set.}
    \label{tab:experiment_algorithms_a2c}
    \\ 
    \hline
    \textbf{Dataset} & \textbf{Annualised return} & \textbf{Cumulative return} & \textbf{Annualised volatility} & \textbf{Sharpe ratio} & \textbf{Max drawdown}  \\ \midrule
    \endfirsthead

    \hline
    \textbf{Dataset} & \textbf{Annualised return} & \textbf{Cumulative return} & \textbf{Annualised volatility} & \textbf{Sharpe ratio} & \textbf{Max drawdown}  \\ \midrule
    \endhead

    \endfoot
    \hline

    \textbf{Dow Jones 30} & 0.2229 & 0.3491 & 0.1483 & 1.4307 & -0.1510 \\ \hline
    \textbf{Euro Stoxx 50} & 0.1467 & 0.2293 & 0.1549 & 0.9617 & -0.1667 \\ \hline
    \textbf{FTSE 100} & 0.1052 & 0.1623 & 0.1268 & 0.8520 & -0.1409 \\ \hline
    \textbf{Commodities} & 0.2353 & 0.3694 & 0.2041 & 1.1372 & -0.1512 \\ \hline
    \textbf{Currencies} & -0.0011 & -0.0017 & 0.0462 & -0.0018 & -0.0665 \\ \hline 
\end{longtable}

For the case of \acrshort{a2c}, the algorithm demonstrates competitive performance particularly for the DowJones30 dataset, achieving a cumulative return of 0.22 and a Sharpe ratio of 1.32. Positive results are also obtained with the commodities dataset, which demonstrates the highest cumulative return of 0.24 and a Sharpe ratio of 1.14. Despite being of the same asset class, the EuroStoxx50 and the FTSE100 datasets show relatively lower performance, with cumulative returns of 0.15 and 0.11, respectively. With regard to the currencies dataset, the performance of the algorithm is less impressive, with a cumulative return of -0.0012 and a Sharpe ratio of -0.0018. 

Similar observations can be made for the other algorithms, although it is worth pointing out that the performance varies significantly across different datasets. \todo{COMPLETE}

Taking the DowJones30 dataset with an environment representation made up of the \acrshort{ohlcv} prices and the indicators, the performance of the algorithms can be benchmarked against traditional strategies and the \acrshort{djia} Index. The evolution of the cumulative returns over the testing period is shown in Figure \ref{fig:dowjones30_indicators_cumulative_returns} and the corresponding performance metrics are summarised in Table \ref{tab:experiment_algorithms_dow30}.

\begin{figure}
    \centering
    \includegraphics[width=\textwidth]{figures/dowjones30_indicators_cumulative_returns.png}
    \caption{Evolution of the Cumulative Returns for the DowJones30 dataset with the \acrshort{ohlcv} prices and indicators environment representation.}
    \label{fig:dowjones30_indicators_cumulative_returns}
\end{figure}

\begin{longtable}{|p{2cm}|p{2.1cm}|p{2.1cm}|p{2.1cm}|p{1.5cm}|p{2cm}|}
    \caption{Algorithm comparison results for the DowJones30 dataset with prices and indicators feature set. The colours correspond to the best performing configurations, with blue for the best performing \acrshort{drl} algorithm and green for the best benchmark.}
    \label{tab:experiment_algorithms_dow30} 
    \\ 
    \hline
    \textbf{Algorithm / Benchmark} & \textbf{Annualised return} & \textbf{Cumulative return} & \textbf{Annualised volatility} & \textbf{Sharpe ratio} & \textbf{Max drawdown}  \\ \midrule
    \endfirsthead

    \hline
    \textbf{Algorithm / Benchmark} & \textbf{Annualised return} & \textbf{Cumulative return} & \textbf{Annualised volatility} & \textbf{Sharpe ratio} & \textbf{Max drawdown}  \\ \midrule
    \endhead

    \endfoot
    \hline

    \textbf{A2C} & 0.2286 & 0.3573 & 0.1557 & 1.3994 & -0.1620 \\ \hline
    \textbf{PPO} & 0.2016 & 0.3134 & 0.1416 & 1.3679 & -0.1370 \\ \hline
    \textbf{DDPG} & 0.2596 & \textcolor{blue}{0.4086} & 0.1558 & \textcolor{blue}{1.5597} & -0.1687 \\ \hline
    \textbf{TD3} & 0.2214 & 0.3456 & 0.1517 & 1.3938 & -0.1609 \\ \hline
    \textbf{SAC} & 0.2050 & 0.3189 & 0.1480 & 1.3340 & -0.1538 \\ \midrule
    \textbf{Mean} & 0.3218 & 0.5114 & 0.1839 & 1.6096 & -0.1983 \\ \hline
    \textbf{Min} & 0.1123 & 0.1705 & 0.1157 & 0.9777 & -0.1066 \\ \hline
    \textbf{Momentum} & 0.3855 & \textcolor{green}{0.6204} & 0.1990 & \textcolor{green}{1.7388} & -0.1929 \\ \hline
    \textbf{Equal} & 0.2066 & 0.3205 & 0.1476 & 1.3461 & -0.1541 \\ \hline
    \textbf{Index} & 0.1110 & 0.1692 & 0.1532 & 0.7635 & -0.1637 \\ \hline
\end{longtable}

The results show that all the algorithms outperform the performance of the index for all the considered metrics, as well as the min-variance portfolio. Out of all the algorithms, \acrshort{ddpg} achieves the highest cumulative return of 0.23 and a Sharpe ratio of 1.49, followed by \acrshort{td3} with a cumulative return of 0.22 and a Sharpe ratio of 1.39. The \acrshort{a2c} algorithm also performs well, while \acrshort{ppo} has the lowest performance with a cumulative return of 0.19 and a Sharpe ratio of 1.24. However, none of these outperform the mean-variance and momentum benchmark strategies, with the latter achieving a Sharpe ratio of 1.74 and a cumulative return of 0.39, which is significantly higher than the performance of the \acrshort{drl} algorithms.

% \section{Experiment: Environment Representation} \label{sec:environment-representation}
% \section{Experiment: Hyper-parameter Tuning} \label{sec:hyper-parameter-tuning}
% \section{Experiment: Reward Function} \label{sec:reward-function}
